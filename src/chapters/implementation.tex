
\chapter{Implementation}

\section{Parser}
\label{parser}
\subsection{Human Readable Syntax}
\label{implHRS}
\subsection{RDF Syntex}
\label{implRDF}

\section{Type Checking}

\section{Evaluation}
\label{implementation_evaluation}

\subsection{Memorisation}
\textbf{NB: tatt fra design delen men ble overflødig og handlet plutselig mer om implementasjonen. Deler kan brukes til dette kampittelet}
\\
When determining whether to use memorisation or not, we have to consider the advantages and disadvantages of utilising memorisation in the context of Frog. On the one hand, memorisation will assumably reduce Frog's time to evaluate the function calls. On the other hand, utilising memorisation in Frog will increase the amount of storage needed. In addition, performing a lookup in the lookup table takes in worst-case O(n)(KILDE) when utilising a hash table. Hence, the lookup time should be less than the time used to evaluate the function call. Consequently, we present two criteria we belives needs to fulfil for memorisation in the context of Frog to be beneficial:
\begin{enumerate}
    \item The assumed time used to perform a lookup should be assumed to be less than the time used to perform the function call.
    \item The assumed storage space needed for memorisation in Frog should be considered not to be remarkably high. 
\end{enumerate}

\para 
Before discussing if Frog should use memorisation, we will introduce two new concepts: function call signature and function call description. 
Firstly, a function call signature is the function name combined with the arguments. Note that the generic arguments are not a part of a function call signature. Hence the function call (fn:plus<<xsd:decimal>> 5 6) and (fn:plus<<xsd:integer>> 5 6) will have a equal function call signature. Secondly, in contrast to the function call signature, the function call description contains the generic arguments, the function name and arguments. Thus the function call (fn:plus<<xsd:decimal>> 5 6) and (fn:plus<<xsd:integer>> 5 6) will not have the same function call description.

\para
To further discuss if memorisation satisfies the two criteria, we must present a possible solution to accomplish memorisation in Frog. Two function calls the same function call signature will evaluate the same value since Frog performs the same calculations on the same arguments. However, function calls with different function call definitions may not have the same type on the evaluated value due to the use of generics. For instance, the function ex:plus2 from figure X will have a different return type which Frog must determine from the generic argument ?t. Hence, the function call (plus2<<xsd:decimal>> 5)  would evaluate to \lstinline{15^^xsd:decimal} while (plus2<<xsd:integer>> 5) would evaluate to \lstinline{15^^xsd:integer}. As illustrated, the evaluated value of two function call with an equal function call signature will be equal. However, the function call description may change the return value type.

\begin{figure}
    \centering
    \begin{minipage}{.5\textwidth}
      \centering
      \includegraphics[scale=0.35]{simpleHash.jpg}
      \caption{A generalisation of the lookup table when only considering the function call signature}
      \label{fig:simpleHash}
    \end{minipage}%
    \begin{minipage}{.5\textwidth}
      \centering
      \includegraphics[scale=0.35]{lookUpGeneral.jpg}
      \caption{A generlaisation of the lookup table consdiering the function call signature and definition}
      \label{fig:lookUpGeneral}
    \end{minipage}
\end{figure}

\para
Thus, a table consisting of the function call signature and the result value will not be enough as it may result in Frog evaluating the value of a not approved type. Evaluating the function call (plus2<<xsd:decimal>> 5) would result in the lookup table shown in figure X. When trying to evaluate (plus2<<xsd:integer>> 5) afterwards Frog will utilise the lookup table since the function call signature is present; however this will result in Frog evaluating the function call to be \lstinline{15^^xsd:decimal}, which is not a valid value. Therefore, we present a solution in which Frog first checks if the function call signature is present and afterwards checks if the function call definition is present. Figure Y illustrates the general structure, while figure Z shows an instance containing the examples mentioned. 

\begin{figure}
    \centering
    \begin{minipage}{.5\textwidth}
      \centering
      \includegraphics[scale=0.35]{lookUpAfterFirst.jpg}
      \caption{Example lookup table only considering the function call signature}
      \label{fig:lookUpAfterFirst}
    \end{minipage}%
    \begin{minipage}{.5\textwidth}
      \centering
      \includegraphics[scale=0.35]{exampelFulllookuptable.jpg}
      \caption{Example lookup table considering the function call signature and defintion}
      \label{fig:exampelFulllookuptable}
    \end{minipage}
\end{figure}

\section{Executor}

